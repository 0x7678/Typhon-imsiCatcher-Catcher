\begin{center}
\textbf{Abstract:}\\
\vspace{.1cm}
\end{center}
For several years now security flaws in the GSM protocol have been known and exploited.
A device called IMSI catcher, first developed in 1996, uses some of these flaws to enable the operator to localise a mobile subscriber and tap into phone calls.
Since only authorities were able to obtain these devices the risk for abuse was deemed minor at first.
However due to the progress in freely available GSM related software and hardware, like OpenBTS and the Universal Software Radio Peripheral, it is now possible for anyone to build an inexpensive version of an IMSI catcher.
Although operation is prohibited by law, the possibility of affordable self-construction increases the risk of abuse in the private sector and in relation with industrial espionage.
Additionally operation is near impossible to discover in retrospect.

The goal of this project is to find means and methods of uncovering IMSI catchers that are active in the close perimeter.
To that end the behaviour of such devices and the differences compared to legitimate base stations will be presented and analysed.
These findings will then be used to implement the IMSI Catcher Detection System, a toolkit with a user friendly graphical interface to gather, analyse and visualise information.
Evaluations against an IMSI catcher shows the effectiveness of the methods used by uncovering several realistic attacks.
The system itself builds upon an open source framework and harvests information about potential IMSI catchers while being invisible itself.
\vspace{.5cm}

\begin{center}
\textbf{Zusammenfassung:}\\
\vspace{.1cm}
\end{center}
Seit einigen Jahren werden bekannte Sicherheitsl\"ucken im GSM Protokoll ausgenutzt um Angriffe durchzuf\"uhren.
Der IMSI-Catcher, ein 1996 entwickeltes Ger\"at, benutzt einige dieser L\"ucken um MobilfunkteilnehmerInnen zu lokalisieren und ihre Anrufe abzuh\"oren.
Da solche Instrumente nur f\"ur Beh\"orden zug\"anglich waren wurde das Missbrauchsrisiko als gering eingesch\"atzt.
Weiterentwicklungen im Bereich frei erh\"altlicher Soft- und Hardware im GSM Bereich, wie etwa OpenBTS oder das Universal Software Radio Peripheral, haben es m\"oglich gemacht einen solchen IMSI-Catcher mit vertretbaren Kosten selbst zu bauen.
Obwohl der Gebrauch solcher Ger\"ate gesetzlich verboten ist, erhöht die Möglichkeit des kostengünstigen Eigenbaus eines IMSI-Catchers das Missbrauchsrisiko im Privatbereich oder im Bereich der Industriespionage enorm.
Erschwerend kommt die Tatsache hinzu, dass der Einsatz kaum nachvollziehbar ist.

Ziel dieses Projektes ist es Vorgehensweisen zu finden, die den Betrieb eines IMSI-Catchers in der Umgebung aufdecken.
Um dies zu erreichen wird das Verhalten eines IMSI-Catchers analysiert und mit dem Verhalten einer legal betriebenen Basisstation verglichen.
Mit Hilfe dieser Ergebnisse wird dann das IMSI-Catcher Detection System entwickelt, ein Programm mit einer benutzerfreundlichen Oberfl\"ache, das dazu dient Informationen zu sammeln, auszuwerten und anzuzeigen.
Auswertungen von Versuchen zum Auffinden echter IMSI-Catcher in verschiedenen realen Angriffsszenarien zeigen die Effektivität  der eingesetzten Methoden.
Das System selbst baut auf einem open source Framework auf, dass es erm\"oglicht Informationen von IMSI-Catchern zu empfangen und dabei selbst unentdeckt zu bleiben.

